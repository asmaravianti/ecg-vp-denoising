I am deeply indebted to my supervisor, Dr. Péter Kovács, for his exceptional mentorship and the opportunity to conduct research on compression-aware ECG denoising within the ELTE Modelling Lab. His decision to entrust me with this TDK thesis project has been a significant milestone in my academic journey, and I am profoundly grateful for this experience.

Throughout the course of this research, Dr. Kovács's insightful guidance, rigorous scientific standards, and extensive expertise in signal processing and machine learning have been fundamental to the development of this work. His weekly supervision sessions, thoughtful critiques, and patient support during challenging experimental phases have been invaluable. I am especially appreciative of his emphasis on methodological precision and scientific integrity, which have profoundly shaped my research approach and analytical thinking.

My sincere thanks go to my family and friends, whose constant encouragement, understanding, and moral support have sustained me throughout this endeavor. Their belief in my capabilities has been a crucial source of strength and motivation.

I also wish to acknowledge the Stipendium Hungaricum Scholarship Programme for the financial assistance that enabled my studies in Hungary. This scholarship has been essential in making this research possible.

