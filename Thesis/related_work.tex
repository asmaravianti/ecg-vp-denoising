\chapter{Related Work}
\label{ch:related}

ECG compression has been extensively studied, with methods ranging from transform-based approaches (DCT, wavelet) to modern deep learning techniques. Traditional methods like SPIHT and JPEG2000 achieve high compression ratios but require separate denoising stages. Recent autoencoder-based approaches enable joint denoising and compression, but most rely on MSE objectives that do not align with clinical quality perception.

Waveform-weighted metrics have been proposed in ECG compression literature, but previous implementations require manual R-peak detection or fixed weighting schemes. Our differentiable WWPRD formulation eliminates this requirement by automatically deriving weights from signal derivatives.

Variable projection methods have shown promise in signal processing for structured representations. We adapt this concept to the ECG domain by replacing the first encoder layer with a learnable projection, potentially improving compression efficiency.

Our work bridges the gap between clinical quality metrics and differentiable training objectives, enabling end-to-end optimization of compression-aware denoising models.

