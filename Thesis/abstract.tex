\chapter*{Abstract}
\addcontentsline{toc}{chapter}{Abstract}

Deploying compression-aware denoising models for electrocardiograms (ECGs) requires optimizing signal fidelity in clinically relevant ways while maintaining controllable bitrates. We propose a single deep 1-D autoencoder that performs joint ECG denoising and compression, trained end-to-end with a differentiable waveform-weighted percent root-mean-square difference (WWPRD) objective instead of the conventional mean-squared error. This objective emphasizes clinically salient waveform regions (QRS complexes, P-waves, T-waves) and aligns optimization with diagnostic perception without requiring manual R-peak detection.

Our model controls the compression ratio through a tunable bottleneck latent dimension and post-training quantization, supporting rate--distortion evaluation over CR $\approx$ 1.4--22.0 (latent dimensions 32 to 2, 4-bit quantization). We introduce quantization-aware training (QAT) using a straight-through estimator to bridge the gap between clean validation metrics and post-quantization performance, reducing quantization degradation from 2.2$\times$ to $<$1.3$\times$. We further propose a variable-projection (VP) front-end that replaces the first encoder convolution with a learnable projection layer to study the effect of structured embeddings on the denoising--compression trade-off.

On all 48 records of the MIT-BIH Arrhythmia dataset with NSTDB noise augmentation, we achieve compression ratios up to 22.0:1 (latent dimension 2 with QAT, 4-bit quantization) with post-quantization PRD 36.20\% and WWPRD 32.23\%, achieving a Quality Score (QS) of 0.6078. Training with WWPRD yields 3--5 dB SNR improvement over MSE baseline at equivalent compression ratios, with better preservation of clinically salient QRS complexes. Our QAT approach enables practical deployment by ensuring models remain robust under quantization constraints, achieving QS $>$ 0.5 which exceeds our target threshold.

Overall, this work demonstrates a practical and fully differentiable framework for training ECG denoisers that are explicitly aware of compression constraints, clarifying how loss-function design, quantization-aware training, and bottleneck capacity influence both diagnostic fidelity and bitrate efficiency in clinical telemetry and wearable monitoring systems.

